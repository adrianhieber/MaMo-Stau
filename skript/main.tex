\documentclass{article}
\usepackage{graphicx}
\usepackage{geometry}
\usepackage{amsmath}
\usepackage{amssymb}
\usepackage{hyperref}
\usepackage[utf8]{inputenc}
\usepackage[ngerman]{babel}
\geometry{
    textheight=23cm,
    textwidth=16.5cm,
    top=1in
}

\setcounter{secnumdepth}{4}

\numberwithin{equation}{section}

% ------------------------------------------------------------------------------
% First Pages
% ------------------------------------------------------------------------------

\title{Mathemathische Modellierung}
\author{Adrian Hieber}
\date{\today}

\begin{document}
\maketitle	
\newpage

\tableofcontents
\newpage

% ------------------------------------------------------------------------------
% Content
% ------------------------------------------------------------------------------

\section{Modulinfos}

Dieses Skript bezieht sich auf den Kurs Mathematische Modellierung von Kräutle. \cite{studon}\\
Es gibt kein Übungsbetrieb, da man hierfür eher an dem Seminar erarbeiten soll.\\
Als Referenzbuch wird Mathematische Modellierung von Eck\cite{eck} empfohlen.


\section{Einführung}

\subsection{Mathematische Modellierung mit gewöhnlichen Diff.gleichungen (ODE):}
Ein sehr einfaches Beispiel aus Populitionsdynamik: Wachstum Schafherde.\\
Zuerst: Welche Größen (unbekannte, Parameter) sind relevant (in physikalischer Dimension):
\begin{equation}
t=Zeit; \qquad \text{y(t)=Anzahl Schafe im Zeitpunkt t}
\end{equation}
\begin{itemize}
    \item Variable t hat die \textbf{Dimension} 'Zeit' und eine Einheit (zb. Tage, Stunden...)
    \item Variable y(t) hat die \textbf{Dimension} 'Anzahl'(='Stück'), kann auch als \textbf{Dimensionslos} bezeichnet werden
\end{itemize}
Für Funktion im mathematischen Modellen sollte man ein Definitionsbereich festlegen, wie z.B. $y:[0,T)\rightarrow\mathbb{R}$, $y[t_0, T]\rightarrow\mathbb{R}$ oder $y:[0,\inf)\rightarrow\mathbb{R}$.

\subsubsection{Bemerkung:}
y ist keine gegebene Funktion, sondern gesucht; a priori ist klar, ob y(t) für beliebige große t existiert.
Existiert eine Lsg für alle Zeite, d.h. $y:[t_0,\inf)\rightarrow\mathbb{R}$ so bezeichnet man sie als \textbf{\underline{globale Lösung}}.

\subsubsection{Modellannahmen:}
Die Wachstumsrate w(t) sei proportional zum aktuellen Bestand: 
\begin{equation}
w(t)=K\cdot g(t)
\end{equation}
Es sei $K\in\mathbb{R}$ (ggf. $K\in\mathbb{R}^+$).
Dabei ist die Wachstums\textbf{rate} definiert als Änderung des Bestands pro Zeitintervall, für 'kurze' Zeitintervalle T, noch genauer für $T\rightarrow 0$:
\begin{equation}
w(t)=\lim_{T\rightarrow 0} \frac{y(t+T)-y(t)}{T}=y'(t)
\end{equation}
Somit sind unsere Modellgleichungen (Anfangswerte nicht vergessen):
\begin{equation}
\boxed{y'(t)=K\cdot y(t), \qquad y(0)=0}
\end{equation}
Mit y'(t) als $\frac{Anzahl}{Zeit}$ und $y(0)$ als Anfangszeitpunkt $t_0=0$.\\
Es enthält 2 \textbf{Parameter} $K\in \mathbb{R}$ (oder $K>0$ bzw $K\geq 0$) $y_0>0$ bzw $y_0\geq 0$ ('Daten')
\paragraph{Dimensionen:}
$y_0$: Anzahl \qquad
$K$: $\frac{1}{Zeit}$ (ergibt sich als Dgl. da y'(t) die Dimension $\frac{Anzahl}{Zeit}$ hat, was sich wieder aus dem Differenzenquotient ergibt.

\subsection{Was haben wir alles vernachlässigt; Aussehen von komplexe Modelle?}
\begin{itemize}
    \item ggf, hängt die Wachstumsrate von Nahrungsangebot ab ('begrenzte Ressourcen') \ref{begrenzte_ressourcen}
    \item  ...
\end{itemize}

\newpage
\section{Entdimensionalisierung}
des Modells $y'(t)=K\cdot y(t), \qquad y(t_0)=y_0$
\begin{equation}
t [Zeit]; \qquad y [Stueck oder Anzahl]
\end{equation}
Wähle dazu (dimensionsbehaftete) "Basisgrößen"(Festlegeung von Maßstäben)$\overline{y}, \overline{t}$\\
definiere:
\begin{equation}
\tau := \frac{t-t_0}{\overline{t}} \quad\text{und}\quad y(\tau):=\frac{y(t)}{\overline{y}}=\frac{y(\overline{t}\tau + t_0)}{\overline{y}}
\end{equation}
Hierbei ist $\tau$ und $y(\tau)$ \underline{\textbf{dimensionslose}} Größen.

\subsection{Modell in diesen Größen ausdrücken:}
bildlich:TODO\\
\begin{equation}
\leadsto
y'(\tau)=\frac{\overline{t}}{\overline{y}} y'(\overline{t}\tau + t_0)
\stackrel{DGL}{=}
\frac{\overline{t}}{\overline{y}}\cdot K \cdot y(\overline{t}\tau + t_0)
=\overline{t} K y(\tau)
\end{equation}
Nicht vergessen: Anfgangsbedingung auch skalieren:\\
Mit $ \tau:=0$ ist $y(0)=\dfrac{y(\overline{t}\cdot 0+t_0)}{\overline{y}}
\stackrel{A.B.}{=}
\dfrac{y(t_0)}{\overline{y}}$ 

\begin{equation}
\leadsto
\boxed{
\begin{aligned}
& y'(\tau)=\overline{t}\cdot K \cdot y(\tau)\\
& y(0)=\frac{y_0}{\overline{y}}
\end{aligned}
}
\end{equation}
\subparagraph{Bemerkung}
Die $y(0)$,$y'(\tau)$, $y(\tau)$ und Kostr. dimensionslos sind, müssen auch die Größen
$\overline{t}K$ und $\frac{y_0}{\overline{y}}$.

\subsubsection{Wahl von $\overline{y}$, $\overline{t}$?}
\begin{itemize}
	\item Größen sind dimensionslos (z.B. $ln\dfrac{y(t)}{y_0}=ln\stackrel{dim. behaftet}{\overbrace{y(t)}}-ln(\stackrel{dim. behaftet}{\overbrace{y_0}})$ ist fragwürdig)
	\item einige Paramter können elemeniert werden
	\subitem Analsis einfacher
	\subitem es sind wenige Simulationen nötig um sich Überblick über die Abhängigkeiten der Lösung von den Paramtern zu verschaffen
	\item Größen haben moderate Werte
	\item Entdimensionalisierung dient als Vorbereitung eines eventuell geplanten \textbf{asymtotischen Entwicklung}\cite{asyEnt}
\end{itemize}






% ------------------------------------------------------------------------------
% Reference and Cited Works
% ------------------------------------------------------------------------------

\newpage
\bibliographystyle{IEEEtran}
\bibliography{References.bib}

% ------------------------------------------------------------------------------
\end{document}
