\documentclass{beamer}
\usepackage[utf8]{inputenc}

\usetheme{Madrid}
\usecolortheme{default}

%------------------------------------------------------------
%This block of code defines the information to appear in the
%Title page
\title[MaMo - Stau] %optional
{MaMo - Stau}

\subtitle{Praxisseminar}

\author[Hieber, Kilic, Rausch, Tatsianda] % (optional)
{Adrian~Hieber \and Cenk~Kilic \and Jakob~Rausch \and Lea~Tatsianda}

\date[\today] % (optional)
{Vorstellung derzeitiger Ergebnisse}

%\logo{\includegraphics[height=1cm]{overleaf-logo}}

%End of title page configuration block
%------------------------------------------------------------



%------------------------------------------------------------
%The next block of commands puts the table of contents at the 
%beginning of each section and highlights the current section:

\AtBeginSection[]
{
  \begin{frame}
    \frametitle{Übersicht}
    \tableofcontents[currentsection]
  \end{frame}
}
%------------------------------------------------------------


\begin{document}

%The next statement creates the title page.
\frame{\titlepage}


%---------------------------------------------------------
%This block of code is for the table of contents after
%the title page
\begin{frame}
\frametitle{Übersicht}
\tableofcontents
\end{frame}
%---------------------------------------------------------


\section{Betrachtung der Faktoren}

%---------------------------------------------------------
%Changing visivility of the text
\begin{frame}
\frametitle{Betrachtung der Faktoren}
Erstmal muss man sich überlegen, welche Faktoren man in Betrachtung sieht in unser Modell einzubauen. 
\newline

\begin{columns}

\column{0.5\textwidth}
\textbf{Modellfaktoren}
\begin{itemize}
    \item<2-> Entfernung zum nächsten Auto
    \item<2-> Geschindigkeit
\end{itemize}

\column{0.5\textwidth}
\textbf{Umgebungsfaktoren}
\begin{itemize}
    \item<3-> Masse $m$
    \item<3-> Länge $l$
    \item<3-> Minimaler Abstand $\epsilon$
    \item<3-> Eine Fahrbahn
    \item<3-> Reaktionszeit $\tau$
    \item<3-> $x_{i}(t) < x_{i-1}(t)+l+\epsilon$
\end{itemize}

\end{columns}
\end{frame}

%---------------------------------------------------------



%---------------------------------------------------------


\end{document}
